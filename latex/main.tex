\documentclass{article}
\usepackage[utf8]{inputenc}
\usepackage{hyperref}
\usepackage{mathtools}
\usepackage{graphicx}
\usepackage{scrextend}
\usepackage{amssymb}

\title{DT057A Project}
\author{raho1501 \& majo1412 \& mafl1400}
\date{May 2019}

\begin{document}

\maketitle

\section{Part I: PRNG and Random Variables} \label{part1}
  \subsection{LCG - Linear Congruential Generator}
    For this project we were tasked to create a LCG\footnote{Linear Congruential Generator}. 
    A LCG uses a recurrence relation to create pseudo random uniformly distributed values, see equation \ref{recrel} for the relation used.
    \begin{equation} \label{recrel}
      X_{n+1} = (aX_{n} + c)\ mod\ m
    \end{equation}
    This equation was then implemented in C++, this is contained in a file called \href{https://github.com/NoRines/simulerings_projekt/blob/master/lcg.h}{\emph{lcg.h}}.
    Using this LCG we were then tasked to generate 1000 values and then test our implementation against the ns3::UniformRandomVariable generator. 
    To test our implementation we were provided with these paramaters:
    \begin{align*}
      a=13 \\
      m=100 \\
      c=1 \\
      seed=1
    \end{align*}
    With these parameters we then generated the 1000 values, and stored the results.
    We then used these same parameters in the ns3::UniformRandomVariable generator, and stored the results.
    We then compared the results from our implementation against the ns3::UniformRandomVariable generator and what we noticed is that our implementation is more cyclic\footnote{Constantly repeating.} than the NS3 implementation. 
    Where our implementation cycles every 20 values or so using the provided parameters.
    With this in mind we then tried to find a set of parameters that gave us better results.
    What we found out is that if $m$ is prime or $m$ is coprime against $a$ we would get a less cyclic generation, rather we would get as many values as possible between 0 and the value $m$.
    So what we can conclude is that the LCG is reliant on the fact that
    \begin{equation}
      (p)\ mod\ n \neq 0,\ p = prime,\ n \in \mathbb{Z}_{+} ,\ n < p
    \end{equation}
    to generate a usable amount of primes.
    Likewise we can conclude that it is enough for $a$ and $m$ to be coprime, as this will give the same results.
    With this in mind we can propose a good scheme to use while choosing parameters, and that scheme will use the fact that:
    \begin{equation}
      m = 2^{x}\ and\ a = 2y+1 \implies m\ and\ a\ is\ coprime \\
    \end{equation}
    
  \subsection{ITM - Inverse Transform Method}
  \subsection{NS3 Normal Distribution Implementation}
  In NS-3 they generate normal distrubuted values using a rejection method called Box-Muller. In NS-3 they generate normal random values by generating two uniform random values u1 and u2 on the range [0,1] and then:
  $$v1 = 2 * u1 - 1$$
  $$v2 = 2 * u2 - 1$$
  $$w = v1^2 + v2^2$$
  $$y = \sqrt{\frac{-2*log(w)}{w}}$$
  $$x1 = mean + v1 * y * \sqrt{variance}$$
  $$x2 = mean + v2 * y * \sqrt{variance}$$
  $$Where\ x1,\ x2 \subseteq N(mean,variance).$$
  
  This polar method that NS-3 uses is a faster rejection method than many ordinary rejection methods, given that you can generate uniform random values at a fast rate while not using trigonometric function. 
  \subsection{Comparison of implementations and distributions}
  \subsection{Conclusions}

\begin{enumerate}
  \item A LCG random generator was created in a file named "lcg.h" for generating random uniformed numbers. The file can be found in the git repo.
  \item 1000 values were generated and transformed onto the range [0,1] by calling the function "norm\_gen()" in the "lcg.h" header. A bulk function was also created if the 1000 values were in an array, then the GPU accelerated function "transformKernel(*input, mod, output,1000)" was called, which can be found in the header "func.cuh".
  \item Our function was compared with the ns3 implementation and the result was that our implementation was worse. Ours were cyclic while the ns-3 wasn't as obviously-cyclic.
  \item The ns3 UniformRandomVariable does not seem to repeat and uses more values between 0 and 1. Using our implementation with the values seed=1, a=13, c=1 and m=100 creates a cycle with 20 values before repetition. Both are however uniformally distributed. Using a m that is a large prime would mean that we get as many values as possible between 0 and the large prime this is because p \% n != 0 for any prime p and value n < p. Howerer if a and p are coprime then this also results in a simmilar way to using primes as m. So if we want to make the modulo operation as effective as possible we can choose a m that is a power of 2. Then if we choose an odd value as a, a and m are coprime.
  \item Combined Multiple-Recursive Generator. The normal random variable in ns3 uses the polar form of the Box-Muller method which is a rejection sampeling method that aviods using trigometric functions. \url{https://en.wikipedia.org/wiki/Box\%E2\%80\%93Muller_transform}
  \item Our implementation is faster we get a time of 3.320s for ns3 it takes 3.806s. We tested with 10 million generated values. With 1 billion generated values our takes ~18.468s while ns3 takes ~1m 7.544s
  \item Acording to the lectures we can use inverse cdf to function on the generated uniform random variable to create a random variable following the desired distriobution. For exponential distribution we can use the function:$$ F^{-1}_x(u) = \frac{-1}{\beta} * ln(1-u) \implies x \in \exp{(\lambda)} $$
  \item When comparing our implementation with the ns3 implementation we get quite simmilar distributions. However our implementation seems have a bit less variance than the ns3 version.
\end{enumerate}


\section{Part II: Mathematical Modelling of a system of Queues} \label{part2}
\begin{figure}[h!]
  \includegraphics[width=\linewidth]{netmap.png}
  \caption{Network topology}
  \label{fig:netmap}
\end{figure}
By inspecting the figure \ref{fig:netmap} we made a Kleinrock approximation to calculate the average number of 
packets in the system. Kleinrock independence approximation assumes that when you have a dense network of queues
the effects of packet streams going into each other results in independent packet arrival-rates as the speed a packet traverses a line is only dependent on the availavble bandwidth of that line and how congested it is, then each
queue can be observed as its own M/M/1 system and be observed isolated from each other. Which simplifies calculations
for the whole system. However its still important to remember that it is still just an approximation and might not
represent reality accuratly but there is proof that it is still a very good approximation that is worth doing. \cite{kenyon2002high}
\begin{table}[h]
\centering
\label{Arrivalrates}
\caption{Arrival rates}
\begin{tabular}{|l|l|l|}
\hline
$\lambda_{ij}$ & Calculation & Arrival rate (Packets/s) \\ \hline
$\lambda_{ae}$ & $\frac{1}{2ms}$  & 500 \\ \hline
$\lambda_{bf}$ & $\frac{1}{2ms}$  & 500 \\ \hline
$\lambda_{cf}$ & $\frac{1}{0.5ms}$  & 2000 \\ \hline
$\lambda_{dg}$ & $\frac{1}{1ms}$  & 1000 \\ \hline
$\lambda_{fg}$ & $\lambda_{bf} + \lambda_{cf}$  & 2500  \\ \hline
$\lambda_{eg}$ & $\lambda_{ae}$ & 500 \\ \hline
$\lambda_{gs}$ & $\lambda_{fg} + \lambda_{dg} + \lambda_{eg}$ & 4000 \\ \hline
\end{tabular}
\end{table}
We then calculated the arrivalrate based on the interarrival times and present them in table \ref{Arrivalrates}. We then calculated the service rates which were based on the available bandwidth on the channel between each node. These were then presented in table \ref{departurerates}.
\begin{table}[h]
\centering
\label{departurerates}
\caption{Departure rates}
\begin{tabular}{|l|l|}
\hline
$\mu_{ij}$ & Service rate(Packets/s)\\ \hline
$\mu_{ae}$ & 6250 \\ \hline
$\mu_{eg}$ & 6250 \\ \hline
$\mu_{bf}$ & 6250 \\ \hline
$\mu_{cf}$ & 6250 \\ \hline
$\mu_{dg}$ & 6250 \\ \hline
$\mu_{fg}$ & 10000 \\ \hline
$\mu_{gs}$ & 12500 \\ \hline
\end{tabular}
\end{table}
The values presented in the tables \ref{Arrivalrates} and \ref{departurerates} were then inserted into the
formula below to calculate the number of packets on each link:
$$N_{ij} = \frac{\lambda_{ij}}{\mu_{ij} - \lambda_{ij} }$$
With the values this becomes the equation-set 1 below:
$$N_{ae} = \frac{\lambda_{ae}}{\mu_{ae} - \lambda_{ae} } \implies \frac{500}{6250 - 500 } = 0.0869565 $$
$$N_{eg} = \frac{\lambda_{eg}}{\mu_{eg} - \lambda_{eg} } \implies \frac{500}{6250 - 500 } = 0.0869565 $$
$$N_{bf} = \frac{\lambda_{bf}}{\mu_{bf} - \lambda_{bf} } \implies \frac{500}{6250 - 500 } = 0.0869565 $$
$$N_{cf} = \frac{\lambda_{cf}}{\mu_{cf} - \lambda_{cf} } \implies \frac{2000}{6250 - 2000 } = 0.470588 $$
$$N_{dg} = \frac{\lambda_{dg}}{\mu_{dg} - \lambda_{dg} } \implies \frac{1000}{6250 - 1000 } = 0.190476 $$
$$N_{fg} = \frac{\lambda_{fg}}{\mu_{fg} - \lambda_{fg} } \implies \frac{2500}{10000 - 2500 } = 0.333333 $$
$$N_{gs} = \frac{\lambda_{gs}}{\mu_{gs} - \lambda_{gs} } \implies \frac{4000}{12500 - 4000 } = 0.470588 $$
$$Equations\ 1.\ Individual\ number\ of\ packets.$$

Later we used the formula below to calculate the average number of packets in the whole system:
$$ \overline{N} = \sum N_{ij} $$
Which gave us the resulting equation-chain:
$$ \overline{N} = \sum N_{ij} = \frac{500}{6250 - 500 } + \frac{500}{6250 - 500 } + \frac{500}{6250 - 500 } + \frac{2000}{6250 - 2000 } + $$ 
$$ \frac{1000}{6250 - 1000 } + \frac{2500}{10000 - 2500 } + \frac{4000}{12500 - 4000 } = 1.638898 packets $$ 


Which concludes the section with the answers to the questions on part 2. The average number of packets are 1.638898
packets and the number of packets per link can be found in equations 1.


\section{Part III: Simulations and Results Comparison} \label{part3}

\subsection{Implementation}
We decided to use the ns3 discrete event network simulator to implement and test the proposed scenario.



\subsubsection{Network topology}
Our network topology consisting of ns3-nodecontainers were interconnected using a point-to
-point helper. The point-to-point helper was used to set the queue-type for all of the
nodes in the network and set the type of queue for all of them. The resulting configuration
for it was five Mbit/s links, two eight Mbit/s links and a ten Mbit/s link where all of
them had a single-server, drop-tail queue. We assumed a infinite queue so this attribute
didn't do much. 

Following this we installed a internet-stack-helper to help with the stack, a
trafic-control-helper to set the queueing discipline and a Ipv4-address-helper to give each
node an ip. 

Lastly we installed the server- and router-applications where the server was a modified
udp-echo-server and the router was a packet-sink which would swallow all packets it
received. 

The topology for the csma version is very similar to the point-to-point version. The only
difference is that the point-to-point-helper was changed to a csma-helper where it got the
same bandwidth values and queue types as the previous version. However this change turned
the channels into half-duplex channels instead of full-duplex as they were previously. This
is due to that they now share a bus to transfer over and if both sides transmit at the same
time, the packets will collide and be dropped.


\subsubsection{Server implementation}
To get the desired behaviour where the server only replies to 70\% of messages and 30\% are sent to the router.
We decided to implement our own version of the UdpEchoServerHelper and UdpEchoServer application.
The new versions of theses classes were called MyUdpEchoServerHelper and MyUdpEchoServer.
When implementing the behaviour first all code in UdpEchoServerHelper and UdpEchoServer was copied into MyUdpEchoServerHelper and MyUdpEchoServer respectively.
We then modified the behaviour by adding the router address as a parameter into the helper class constructor.
This address was stored in a private member variable called m\_routerAddress.
This address needed to be sent to the MyUdpEchoServer application that the helper creates.
So in the private install function of the helper class we use the object factory to create a pointer to a MyUdpEchoServer object then the router address is set in this object by calling a setter function.
The application is then added to the node that the helper is installing.

The modifications made to the MyUdpEchoServer class is first that two private member variables were added one for the router address called m\_routerAddress and one pointer for the uniform random variable called randVar.
The router address is set by the helper using a setter function and the random variable is set in the class constructor using the create object function.
To change the behaviour the only other function that needed modification was HandleRead.
In this function we first generate a random variable using the uniform random variable.
Then if the value is larger then 0.7 a bool is set to true.
This bool variable decides if the message should go to the router or not.
When the response packet is sent the bool variable is checked and the router address i applied if true and the sender address is applied if false.

\subsubsection{Scheduling}
To schedule sending events for the different sending node A, B, C and D the simulator function called ScheduleWithContext was used.
This function takes a callback function that generates new data by first randomizing the packet size with an exponential random variable and then sending the packet.
After the packet is sent a new call to the traffic generator function is scheduled using the simulator's ScheduleWithContext.
The time to the next packet is set to follow an exponential random variable.
This results in the nodes sending packets according to a Poisson process.
This means that none of the sending nodes have any applications installed instead the sending behaviour is directly handled by the simulator.
First we considered using an OnOffHelper to send packets where the off time was exponentially distributed but we could not find any way to make it use our random number generator.
We also considered creating our own version of the UdpEchoClient application but this would be much more time consuming than just directly handling the packet sending with the simulator, which is what we decided to use.

\subsubsection{Logging method}
The type of data we wanted to be able to see when logging was the queue lengths for each connection and delay for each flow of data.
We also used pcap to log the packets for some nodes for debugging purposes.

To log the number of packets that were in each queue we used the simulator to schedule calls to a logging function.
This function took as parameters the queue container that keeps the status for each queue as well as a file stream.
The number of packets in each queue was then logged as comma separated values which made it very easy to take the outputted file and do analysis on it using python.

To get information on the delay values of each flow of data we used the ns3 flow monitor tool to output an xml containing all the flow information.
We could then read that file and see the total delay for all flows.
This could then be used to calculate the average delay on a specific link.

\subsection{Simulation}
When simulating the final setup we compared it to the calculated expectations. For the average 
number of packets between node G and the server was calculated to be around 0.47 packets, 
however the simulated result was around 0.064 packets which is significantly smaller than the 
expected calculation deems. 

The average queuing delay and total average delay for the packets traversing the link F to G 
was aquired by creating a flow-monitor statistical analyser file. From that resulting XML-file 
we calculated the delay by taking the delaysum for the flow. We used the total delay values for 
the nodes that is presented just below.
$$\frac{((b \rightarrow g) - (b \rightarrow f)) + ((c \rightarrow g) - (c \rightarrow f))}{number\ of\ packets}$$
Which resulted with the delay from F to G to be about 0.123 milliseconds.  
When observing the delay between A and the server we got that it was around 0.454 milliseconds 
and the reversed path (server to A) was around 0.478 milliseconds. 

Later we changed the point-to-point configuration to a CSMA-based one. This had change did not 
have any significant effect on the data rate or the delay as the differences were negliable.
However the number of packets in queue at the server to G queue was huge.

Finally we swaped the NS-3 random variables to our implemented versions. When we did this we
barely noticed any difference. We almost got the same results as before except abit lower.
The queue size form G to server was now 0.055 and the size was on F to G was now 0.150. 
The Delay from F to G was unchanged, the delay from the server to A was 0.46 which is slightly 
lower than before and from A to the server was now 0.48 . These are however within the bounds 
of deviations so we can say that they are equal.


\subsubsection{Evaluation}
The difference in number of packets between G and the server is of a noticable size. We have 
two theories around why that is the case.

\textbf{First} the calculations use the Kleinrock independence approximation which may not be 
a good approximation for the system. This is because the system has quite low traffic while the 
Kleinrock approximation works better for high traffic systems. We tested halving all the data 
rates in the system and got results more close to the calculated values.

\textbf{Second} the simulation might not take into account the packets that are in service 
by only counting the packets that are in queue. This might be responsible for part of the 
error since the Kleinrock approximation calculates the total number of packets in the point 
to point connection. When measuring the average queue size between g and the router the 
results are 0.This is probably because the router is able to consume the packet fast enough 
so that no queue has time to form.

When we changed to CSMA on busses we got a larger queue at the server to G. This is probably due
to the fact that we have the combined transmittions from G redirected from D, E and F which are 
all heading to the server, which the return packets has to fight for space against. And since the
busses are only half-duplex and can only send in one direction at a time there will be packet 
collisions on that bus, which adds more and more packets at the server to G backlog for retransmittions.
This is represented in the image below where the large packet streams are heading towards G and a bunch 
of slower responses are heading back from the server which they have to fight for space against, which 
leads to a backlog forming at G for the responses.
\begin{figure}[h!]
  \includegraphics[width=\linewidth]{csmamap.png}
  \caption{CSMA flow map}
  \label{fig:csma}
\end{figure}

\section{Conclusions}

\bibliographystyle{unsrt}
\bibliography{ref}
\end{document}